\documentclass{beamer}
\usepackage{amsmath, graphicx, amssymb, pstricks}
\author{Akriti Sengar \\ $2020705$}
\title{\textbf{Aryabhatta}}
\institute{Shaheed Rajguru College of Applied Sciences for Women \\ University of Delhi, Delhi:- $110096$, India}
\usetheme{Warsaw}
\usefonttheme{structurebold}
\begin{document}
	\maketitle
	\begin{frame}
		\begin{center}
\includegraphics[width=8cm,height=8cm]{arya}
		\end{center}
	\end{frame}
	
	\begin{frame}{Aryabhatta}
		\begin{block}{}
		\textbf{Aryabhata} (ISO: Āryabhaṭa) or \textbf{Aryabhata I} ($476–550$ CE) was the first of the major mathematician-astronomers from the classical age of Indian mathematics and Indian astronomy. \pause His works include the Āryabhaṭīya (which mentions that in $3600$ Kali Yuga, $499$ CE, he was $23$ years old) and the Arya-siddhanta.
		\end{block}
	\end{frame}

\begin{frame}
\begin{block}{Education}
	At some point, he went to Kusumapura for advanced studies and lived there for some time. \pause Both Hindu and Buddhist tradition, as well as Bhāskara I (CE $629$), identify Kusumapura as Pāṭaliputra, modern Patna. \pause Aryabhata was the head of an institution (kulapa) at Kusumapura, and, because the university of Nalanda was in Pataliputra at the time and had an astronomical observatory, it is speculated that Aryabhata might have been the head of the Nalanda university as well. \pause Aryabhata is also reputed to have set up an observatory at the Sun temple in Taregana, Bihar.
\end{block}
\end{frame}

\begin{frame}
	\begin{block}{Works}
				The Arya-siddhanta, a lost work on astronomical computations. \pause This work appears to be based on the older Surya Siddhanta and uses the midnight-day reckoning, as opposed to sunrise in Aryabhatiya. \pause It also contained a description of several astronomical instruments: the gnomon (shanku-yantra), a shadow instrument (chhAyA-yantra), possibly angle-measuring devices, semicircular and circular (dhanur-yantra / chakra-yantra), a cylindrical stick yasti-yantra, an umbrella-shaped device called the chhatra-yantra, and water clocks of at least two types, bow-shaped and cylindrical. \pause \\
		His major work, Aryabhatiya, a compendium of mathematics and astronomy, was extensively referred to in the Indian mathematical literature and has survived to modern times. \pause The mathematical part of the Aryabhatiya covers arithmetic, algebra, plane trigonometry, and spherical trigonometry. It also contains continued fractions, quadratic equations, sums-of-power series, and a table of sines.
	\end{block}
\end{frame}

\begin{frame}
	\begin{block}{Place value system and zero}
		The place-value system, first seen in the $3^{\text{rd}}$-century Bakhshali Manuscript, was clearly in place in his work. While he did not use a symbol for zero, the French mathematician Georges Ifrah argues that knowledge of zero was implicit in Aryabhata's place-value system as a place holder for the powers of ten with null coefficients.
\end{block}
\end{frame}
\end{document}