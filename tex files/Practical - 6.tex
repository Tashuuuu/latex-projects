\documentclass{article}
\usepackage{amsmath,amssymb,pstricks}
\title{\textbf{Practical - 6}\\P.S. Tricks}
\author{Akriti Sengar \\ $2020705$}
\date{September $30^{\text{th}}$, 2021}
\begin{document}
	\maketitle
	\noindent \textbf{Q1. Examples}
	\begin{enumerate}
		\item[\textbf{10.1.}] Let's start with a little figure displaying lines and circles. \\
		\begin{pspicture}(6,5.4)
			\psline(1,4)(1,1)(5,1)(1,4)
			\pscircle[fillstyle=solid,fillcolor=pink](2,2){1}
			\pscircle[linestyle=dotted,linewidth=0.1](3,2.5){2.5}
		\end{pspicture} \\
		
		\item[\textbf{10.2.}] We draw a figure illustrating the straightedge and compass construction of an equilateral triangle. \\
		\begin{pspicture}(6,6)
			\psline(1,2)(5,2)(3,5.464)(1,2)
			\psarc[linewidth=0.01](1,2){4}{0}{70}
			\psarc[linewidth=0.01](5,2){4}{110}{180}
		\end{pspicture}
		
		\item[\textbf{10.3.}] We draw a picture of a circle with a shaded sector, and some symbols and an equation in math mode. \\
		\begin{pspicture}(5,5)
			\pscircle(2,2){2}
			\pswedge[fillstyle=solid,fillcolor=lightgray](2,2){2}{0}{70}
			\put(2.25,2.17){$\theta$}
			\put(2.8,1.7){$r$}
			\put(3.8,3.3){$A=r\theta$}
		\end{pspicture}
		
		\item[\textbf{10.4.}] We draw an ellipse with a shaded sector. \\
		\begin{pspicture}(5,5)
			\psclip{\psellipse(2.5,2.5)(3.5,2)}
			\psline[fillstyle=solid,fillcolor=lightgray](2.5,2.5)(7,2.5)(7,7)(2.5,2.5)
			\psellipse(2.5,2.5)(3.5,2)
			\endpsclip
		\end{pspicture}
		
		\item[\textbf{10.5.}] We create a picture of a parabola together with its focus and directrix. We add some labels and an equation using math mode. \\
		\begin{pspicture}(3,3)
			\parabola{<->}(3.7,1.2)(6,0) 
			\psline(6,1.3)(7.3,0.37)
			\psline[linewidth=0.04](7.3,0.37)(7.3,-1.2)
			\psline[linewidth=0.04]{<->}(3.7,-1.2)(8.5,-1.2)
			\put(6,1.3){\circle*{0.09}}
			\put(7.3,0.37){\circle*{0.09}}
			\put(7.3,-1.2){\circle*{0.09}}  
			\put(5.5,1.3){$F$}
			\put(7.3,0.6){$P$}
			\put(7.3,-1.7){$D$}
			\put(7.8,1.6){$PF=PD$}
		\end{pspicture}
\newpage
		\item[\textbf{10.6.}] We make a tessellation pattern of crosses.\\
		\begin{pspicture}(10,7)
			\psset{unit=0.3}
			\def \twelve {\psline[fillstyle=solid,fillcolor=gray](5,9)(6,9)(6,8)(7,8)(7,7)(6,7)(6,6)(5,6)(5,7)(4,7)(4,8)(5,8)(5,9)}
			\put(2,1){\twelve}
			\put(1,4){\twelve}
			\put(0,7){\twelve}
			\put(5,2){\twelve}
			\put(4,5){\twelve}
			\put(3,8){\twelve}
			\put(8,3){\twelve}
			\put(7,6){\twelve}
			\put(6,9){\twelve}
			\put(11,4){\twelve}
			\put(10,7){\twelve}
			\put(9,10){\twelve} 
		\end{pspicture}    		
	\end{enumerate}

	\noindent \textbf{Q2. Make a picture of a 5-12-13 Pythagorean triangle, as below.}\\
	\begin{pspicture}(12,8)
		\psline[linewidth=0.09cm](2,1)(11,1)(11,4.5)(2,1)
		\psline[linewidth=0.09cm](11,1.3)(10.7,1.3)(10.7,1)
		\put(7,0.2){\huge$12$}
		\put(11.3,2.2){\huge$5$}
		\put(6.2,3.5){\huge$13$}
	\end{pspicture}
\newpage
	\noindent \textbf{Q3. Make a diagram kike the one that follows, illustrating Archimedes' demonstration of the formula for the area of a circle.}\\
		\begin{pspicture}(10,6)
		\pscircle(2,3){2}
		\psline(2,3)(4,3)
		\psline(2,3)(2,5)
		\psline(2,3)(0,3)
		\psline(2,3)(2,1)
		\psline(2,3)(3.78,3.9)
		\psline(2,3)(3.05,4.7)
		\psline(2,3)(0.95,4.7)
		\psline(2,3)(0.2,3.9)
		\psline(2,3)(0.2,2.1)
		\psline(2,3)(0.95,1.3)
		\psline(2,3)(3.05,1.3)
		\psline(2,3)(3.78,2.1)
		\put(3,2.7){$r$}
		\psline(5,3)(5.5,5)
		\psline[linewidth=0.05cm](5.5,5)(6,3)(6.5,5)(7,3)(7.5,5)(8,3)(8.5,5)(9,3)(9.5,5)(10,3)(10.5,5)(11,3)
		\psline(11,3)(11.5,5)
		\psline[linewidth=0.07cm](12,3)(12,5)
		\psline[linewidth=0.07cm](5,2.5)(11,2.5)
		\put(12.3,4){$r$}
		\put(7.5,1.9){$\sim\pi r$}
		\def \y {\psarc(0,0){2.1}{75.5}{105}}
		\put(6,3){\y}
		\put(7,3){\y}
		\put(8,3){\y}
		\put(9,3){\y}
		\put(10,3){\y}
		\put(11,3){\y}
		\def \z {\psarc(0,0){2.1}{255}{285}}
		\put(5.5,5){\z}
		\put(6.5,5){\z}
		\put(7.5,5){\z}
		\put(8.5,5){\z}
		\put(9.5,5){\z}
		\put(10.5,5){\z}
		\put(7,1){$A=(\pi r)r=\pi r^2$}
	\end{pspicture}
	
	\noindent \textbf{Q4. Make a picture of the Venn diagram below.}\\
		\begin{pspicture}(6,6)
		\pscircle(4,3){2}
		\pscircle(6,3){2}
		\put(2.3,3.3){multiples}
		\put(2.7,2.8){of $2$}
		\put(6.3,3.3){multiples}
		\put(6.7,2.8){of $3$}
		\psarc[fillstyle=solid,fillcolor=lightgray](4,3){2}{300}{60}
		\psarc[fillstyle=solid,fillcolor=lightgray](6,3){2}{120}{240}
		\put(4.3,3.3){multiples}
		\put(4.7,2.8){of $6$}
	\end{pspicture}
\newpage	
	\noindent \textbf{Q5. Make a picture of an ellipse and its two foci $F_1$ and $F_2$, illustrating the relation $F_1P+F_2P= constant$, where $P$ is a point on the ellipse.}\\ 
		    	\begin{center}
	\begin{pspicture}(4,4)
	\psline{<->}(-4,0)(4,0)
	\psline{<->}(0,4)(0,-4)
	\psellipse(0,0)(3.5,2)
	\psline[linewidth=0.06](1.75,1.75)(3,0)
	\psline[linewidth=0.06](1.75,1.75)(-3,0)
	\pscircle[fillstyle=solid,fillcolor=black](-3,0){0.09}
	\pscircle[fillstyle=solid,fillcolor=black](3,0){0.09}
	\pscircle[fillstyle=solid,fillcolor=black](1.75,1.75){0.09}
	\put(-3.2,-0.5){$F1$}
	\put(2.8,-0.5){$F2$}
	\put(2,2){$P$}
	\put(4,3){$F_1P+F_2P=constant$}
\end{pspicture}
	\end{center}
\end{document}