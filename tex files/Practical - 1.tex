\documentclass{article}
\usepackage{amsmath}
\title{\textbf{Practical - 1} \\ Basic Mathematical equation}
\author{Akriti Sengar \\ $2020705$}
\date{September $2^{\text{nd}}$, 2021}
\begin{document}
	\maketitle
	
	\noindent \textbf{Q1. Typeset the following.}
	\begin{enumerate}
		\item Suppose that $x=137$.
		\item Let $n=3.$ Then $n^2+1=10$.
		\item The curve $y=\sqrt{x}$, where $x \geq 0$, is concave downward.
		\item If $\sin \theta=0$ and $0 \leq \theta < 2\pi$, then $\theta=0$ or $\theta=\pi$.
		\item It is not always true that \\ \[\frac{a+b}{c+d}=\frac{a}{c}+\frac{b}{d}.\]
	\end{enumerate}
	
	\noindent \textbf{Q2. Make the following equations.}
	\begin{enumerate}
		\item \[3^3+4^3+5^3=6^3\]
		\item \[\sqrt{100}=10\]
		\item \[(a+b)^3=a^3+3a^2b+3ab^2+b^3\]
		\item \[\sum\limits_{k=1}^{n} k=\frac{n(n+1)}{2}\]
		\item \[\frac{\pi}{4}=\frac{1}{1}-\frac{1}{3}+\frac{1}{5}-\frac{1}{7}+\frac{1}{9}-\frac{1}{11}+\cdots\]
		\item \[\cos \theta=\sin(90^\circ-\theta)\]
		\item \[e^{i \theta}=\cos \theta+i \sin \theta\]
		\item \[\lim\limits_{\theta \to 0} \frac{\sin \theta}{\theta}=1\]
		\item \[\lim\limits_{x \to \infty} \frac{\pi (x)}{x/\log x}=1\]
		\item \[\int_{-\infty}^{\infty} e^{-x^2} dx=\sqrt{\pi}\]
	\end{enumerate}
	
	\noindent \textbf{Q3. Typeset the following sentences.}
	\begin{enumerate}
		\item Positive numbers $a, b,$ and $c$ are the side lengths of a triangle if and only if $a+b>c, b+c>a$, and $c+a>b$.\\
		\item The area of a triangle with side lengths $a, b, c$ is given by \textit{Heron's formula}: \\ \[A = \sqrt{s(s-a)(s-b)(s-c)},\] \\ where $s$ is the semi-perimeter $(a+b+c)/2$.\\
		\item The volume of a regular tetrahedron of edge length $1$ is $\sqrt{2}/12$.\\
		\item The quadratic equation $ax^2+bx+c=0$ has roots \\ \[r_1,r_2=\frac{-b \pm \sqrt{b^2-4ac}}{2a}.\]\\
		\item The \textit{derivative} of a function $f$, denoted $f'$, is defined by \\ \[f'(x)=\lim_{h \to 0} \frac{f(x+h)-f(x)}{h}.\]\\
		\item A real-valued function $f$ is convex on an interval $I$ if \\ \[f(\lambda x+(1-\lambda)y) \leq \lambda f(x)+(1-\lambda)f(y),\] \\ for all $x,y \in I$ and $0 \leq \lambda \leq 1$.\\
		\item The general solution to the differential equation \\ \[y''-3y'+2y=0\] \\ is \\ \[y=C_1e^x + C_2e^{2x}.\]\\
		\item The \textit{Fermat number} $F_n$ is defined as \\ \[F_n=2^{2^n}, \;\;\;\; n \geq 0.\]
	\end{enumerate}
	
\end{document}