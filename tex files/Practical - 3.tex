\documentclass{article}
\usepackage{amsmath}
\title{\textbf{Practical - 3} \\ Multi-line Expression}
\author{Akriti Sengar \\ $2020705$}
\date{September $13^{\text{rd}}$, 2021}
\begin{document}
	\maketitle
	\noindent \textbf{Q1. Typeset the following}
	\begin{eqnarray*}
		e^x & = & \sum\limits_{n=0}^{\infty}\frac{x^n}{n!}\\
		e^x & = & \frac{x^0}{0!}+ \frac{x^1}{1!}+ \frac{x^2}{2!}+ \frac{x^3}{3!}+ \cdots\\
		e^{-1} & = & \frac{({-1})^0}{0!}+ \frac{({-1})^1}{1!}+ \frac{({-1})^2}{2!}+ \frac{({-1})^3}{3!}+ \cdots\\
	\end{eqnarray*}
	
	\noindent \textbf{Q2. Make the following multi-line equation.}
	\begin{eqnarray*}
		1+2 & = & 3\\
		4+5+6 & = & 7+8\\
		9+10+11+12 & = & 13+14+15\\
		16+17+18+19+20 & = & 21+22+23+24\\
		25+26+27+28+29+30 & = & 31+32+33+34+35\\
	\end{eqnarray*}
	
	\noindent \textbf{Q3. Make the following multi-line equation.}
	\begin{eqnarray*}
		(a+b)^2 & = & (a+b)(a+b)\\
		& = & (a+b)a+(a+b)b\\
		& = & a(a+b)+b(a+b)\\
		& = & a^2+ab+ba+b^2\\
		& = & a^2+ab+ab+b^2\\
		& = & a^2+2ab+b^2 
	\end{eqnarray*}
	
	\noindent \textbf{Q4. Make the following multi-line equation.}
	\begin{eqnarray*}
		\tan(\alpha+\beta+\gamma) & = & \frac{\tan(\alpha+\beta)+\tan\gamma}{1-\tan(\alpha+\beta)\tan\gamma}\\\\
		& = & \frac{\frac{\tan\alpha+\tan\beta}{1-\tan\alpha\tan\beta}+\tan\gamma}{1-\left( \frac{\tan\alpha+\tan\beta}{1-\tan\alpha\tan\beta} \right)\tan\gamma}\\\\
		& = & \frac{\tan\alpha+\tan\beta+(1-\tan\alpha\tan\beta)\tan\gamma}{1-\tan\alpha\tan\beta-(\tan\alpha+\tan\beta)\tan\gamma}\\\\
		& = & \frac{\tan\alpha+\tan\beta+\tan\gamma-\tan\alpha\tan\beta\tan\gamma}{1-\tan\alpha\tan\beta-\tan\alpha\tan\gamma-\tan\beta\tan\gamma}
	\end{eqnarray*}
	
	\noindent \textbf{Q5. Make the following multi-line equation.}
	\begin{eqnarray*}
		\prod_p \left(1-\frac{1}{p^2}\right) & = & \prod_p \frac{1}{1+\frac{1}{p^2}+\frac{1}{p^4}+\cdots}\\\\
		& = & \left(\prod_p \left( 1+\frac{1}{p^2}+\frac{1}{p^4}+\cdots\right) \right)^{-1}\\\\
		& = & \left( 1+\frac{1}{2^2}+\frac{1}{3^2}+\frac{1}{4^2}+\cdots\right)^{-1}\\\\
		& = & \frac{6}{\pi^2}
	\end{eqnarray*}
	
	\noindent \textbf{Q6. Make the following multi-line equation.}
	\begin{eqnarray*}
		\log(1+x) & = & x - \frac{x^2}{2} + \frac{x^3}{3} - \frac{x^4}{4} + \cdots\\\\
		\frac{x^2}{2} & = & \pi x - 2\sum_{n=1}^{\infty} \frac{1}{n^2} +2\sum_{n=1}^{\infty} \frac{\cos x}{n^2} \quad\quad\quad 0\leq x\leq2\\\\
		& = & \frac{4\pi^2}{3}+4\sum_{n=1}^{\infty} \frac{\cos x}{n^2}-4\pi \sum_{n=1}^{\infty} \frac{\sin x}{n} \quad\quad\quad 0<x<2\pi\\\\
		\int_{0}^{x} f(\theta) d\theta & = & \frac{a_0x}{2}+\sum_{n=1}^{\infty} \frac{b_n}{n}+\sum_{n=1}^{\infty} \left( \frac{a_n}{n} \sin nx -\frac{b_n}{n}\cos nx \right)
	\end{eqnarray*}
	
\end{document}