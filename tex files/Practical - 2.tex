\documentclass{article}
\usepackage{amsmath}
\title{\textbf{Practical - 2} \\ Derivatives and array}
\author{Akriti Sengar \\ $2020705$}
\date{September $7^{\text{th}}$, 2021}
\begin{document}
	\maketitle
	\noindent \textbf{Q1. Make the following equation.}
	\[ \left( \frac{a+b}{x+y} \right)^{\frac{1}{3}}\]
	
	\noindent \textbf{Q2. }
	\begin{enumerate}
		\item Let $\mathbf{x} = (x_1,\dots,x_n) $, where the $x_i$ are non-negative real numbers. Set \[ M_r(\mathbf{x}) = \left( \frac{x_1^r+x_2^r+\cdots +x_n^r}{n} \right)^{1/r} , \; \; r \in \mathbf{R} \backslash \{0\}, \] and \[ M_0(\mathbf{x}) = (x_1x_2\dots x_n)^{1/n}. \] We call $M_r(\mathbf{x})$ the \textit{rth power mean} of $\mathbf{x}.$ \\
		Claim: \[ \lim\lim\limits_{r \to 0}  M_r(\mathbf{x}) = M_0(\mathbf{x}). \]
		\item Define \[V_n = \left[ \begin{array}{ccccc}
			1 & 1 & 1 & \cdots & 1 \\ 
			x_1 & x_2 & x_3 & \cdots & x_n \\ 
			x_1^2 & x_2^2 & x_3^2 & \cdots & x_n^2 \\
			\vdots & \vdots & \vdots & \ddots & \vdots \\
			x_1^{n-1} & x_2^{n-1} & x_3^{n-1} & \cdots & x_n^{n-1}
		\end{array} \right]. \] 
		We call $V_n$ the \textit{Vandermonde matrix} of order $n$. \\
		Claim: \[ \text{det } V_n = \prod\limits_{1 \leq i < j \leq n} (x_j-x_i). \] 
	\end{enumerate}
	
	\noindent \textbf{Q3. Make the following equations. Notice the large delimiters.}
	\begin{enumerate}
		\item \[\frac{d}{dx} \left( \frac{x}{x+1} \right) = \frac{1}{(x+1)^2}\]
		\item \[\lim\limits_{n \to \infty} \left( 1+\frac{1}{n} \right)^n = e\]
		\item \[ \left| \begin{array}{cc}
			a & b \\ c & d 
		\end{array} \right| = ab-bc \]
		\item \[ R_\theta  = \left[ \begin{array}{cc}
			\cos \theta & -\sin \theta \\ 
			\sin \theta & \; \; \; \cos \theta
		\end{array} \right] \]
		\item \[ 
		\left| \begin{array}{ccc}
			\mathbf{i} & \mathbf{j} & \mathbf{k} \\ 
			a_1 & a_2 & a_3 \\ b_1 & b_2 & b_3
		\end{array} \right| = \left| \begin{array}{cc} 
			a_2 & a_3 \\ b_2 & b_3
		\end{array} \right| \mathbf{i} - \left| \begin{array}{cc} 
			a_1 & a_3 \\ b_1 & b_3
		\end{array} \right| \mathbf{j} + \left| \begin{array}{cc} 
			a_1 & a_2 \\ b_1 & b_2
		\end{array} \right| \mathbf{k}
		\]
		\item \[ \left[ \begin{array}{cc}
			a_{11} & a_{12} \\ a_{21} & a_{22} 
		\end{array} \right] \left[ \begin{array}{cc}
			b_{11} & b_{12} \\ b_{21} & b_{22} 
		\end{array} \right] = \left[ \begin{array}{cc}
			a_{11}b_{11}+a_{12}b_{21} & a_{11}b_{12}+a_{21}b_{22} \\
			a_{21}b_{11}+a_{22}b_{21} & a_{21}b_{12}+a_{22}b_{22} 
		\end{array} \right] \]
		\item \[ f(x) = \begin{cases}
			-x^2, \quad x<0 \\ 
			x^2, \quad 0 \leq x \leq 2 \\
			4, \quad x>2
		\end{cases} \]
	\end{enumerate}
	
	\noindent \textbf{Q4. Type the following matrix using matrix environment.}
	\[ M_\phi = \begin{bmatrix}
		\hat \phi_{11} & \hat \phi_{12} & \cdots & \hat \phi_{1n} \\
		\xi_{121} & \xi_{122} & \cdots & \xi_{12n} \\
		x_{31} & x_{32} & \cdots & x_{3n} \\ 
		\vdots & \vdots &   & \vdots \\
		n_{121}  & n_{122} &   & n_{12n}
	\end{bmatrix} \]
	
\end{document}