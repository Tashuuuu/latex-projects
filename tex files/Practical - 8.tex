\documentclass{beamer}
\usepackage{amsmath, graphicx, amssymb, pstricks}
\author{Akriti Sengar \\ $2020705$}
\title{Permutation and Combination}
\institute{Shaheed Rajguru College of Applied Sciences for Women \\ University of Delhi, Delhi:- $110096$, India}
\usetheme{Warsaw}
\usefonttheme{structurebold}
\titlegraphic{\includegraphics[width=2cm,height=2cm]{Logo}}
\begin{document}
	\maketitle
	\begin{frame}{What this presentation covers?}	
		\begin{enumerate}
			\item What is combination?
			\item Formula for combination.
			\item Example of combination.
			\item What is permutation?
			\item Formula for permutation.
			\item Example of permutation.
			\item How is combination and permutation different?
			\item Similarity in them.
		\end{enumerate}
	\end{frame}
	
	\begin{frame}\frametitle{What is combination?}
		\begin{block}{}
		Combination is a mathematical technique that determines the number of possible arrangements in a collection of items where the order of the selection does not matter.
		\end{block}
	\end{frame}
	
	\begin{frame}{Formula for Combination.}
		\begin{block}{}
		Mathematically, determining the number of possible arrangements with no repetition is expressed as: \pause
		\end{block}
		\includegraphics[width=4cm,height=1cm]{c,n,k}
		\begin{itemize}
			\item[>] n – total number of elements in a set.\\
			\item[>] k – number of selected objects.\\
			\item[>] ! – factorial.
		\end{itemize}\pause
		\begin{block}{Factorial}
			Denoted by “!” is a product of all positive integers less or equal to the number preceding factorial sign. For example, $3! = 1 x 2 x 3 = 6$. \\
		\end{block}\pause
		\begin{block}{NOTE:}
			Above formula can be used only when the objects from a set are selected without repetition.
		\end{block}
	\end{frame}
	
	\begin{frame}{Example of Combination.}
		\begin{enumerate}
			\item \textbf{Ques:} Choosing $3$ desserts from a menu of $10$. \\ \textbf{Ans:} \[C(10,3) = 120. \] \\
			\item \textbf{Ques:} A fruit salad is a combination of apples, grapes and bananas. We don't care what order the fruits are in, they could also be "bananas, grapes and apples" or "grapes, apples and bananas", its the same fruit salad. \\ \textbf{Ans:} "The combination to the safe is $472$".
		\end{enumerate}
	\end{frame}
	
	\begin{frame}{What is permutation?}
		\begin{block}{}
		Permutation is a mathematical technique that determines the number of possible arrangements in a set when the order of the arrangements matters.\\ Common mathematical problems involve choosing only several items from a set of items with a certain order.
		\end{block}
	\end{frame}
	
	\begin{frame}{Formula for permutation.}
		\begin{block}{}
		The general permutation formula is expressed in the following way:
		\end{block}
		\includegraphics[width=4cm,height=1cm]{p,n,k}
		\begin{itemize}
			\item[>] n – total number of elements in a set.\\
			\item[>] k – number of selected elements arranged in a specific order.\\
			\item[>] ! – factorial.
		\end{itemize}\pause
		\begin{block}{NOTE:}
			Above formula is used in situations when we want to select only several elements from a set of elements and arrange the selected elements in a special order.
		\end{block}
	\end{frame}
	
	\begin{frame}{Example of permutation.}
		\begin{enumerate}
			\item Permutation of set $A = \{1,6\}$ is $2$, such as $\{1,6\}, \{6,1\}$. As you can see, there are no other ways to arrange the elements of set $A$.
			\item \textbf{Ques:} If we choosing $3$ desserts from a menu of in order, then it's permutation will be? \\ \textbf{Ans:} \[P(10,3) = 720\]
		\end{enumerate}
	\end{frame}
	
	\begin{frame}{How is combination and permutation different?}
		\begin{block}{}
		Permutations and combinations are mathematical techniques that are frequently confused with one another. \pause \\ However, in combinations, the order of the chosen items does not influence the selection but in permutations, the order of the selected items is essential.
		\end{block}
 \pause
 \begin{block}{}
 For example, the arrangements $ab$ and $ba$ are equal in combinations (considered as one arrangement), while in permutations, the arrangements are different.
 \end{block}
\pause
		\begin{block}{Permutation and combination}
			When we select the data or objects from a certain group, it is said to be permutations, whereas the order in which they are represented is called combination.
		\end{block}
	\end{frame}
	
	\begin{frame}
		\begin{tabular}{|l|l|} \hline
			\bf{Permutation} & \bf{Combination} \\ \hline
			Arranging people, digits, etc. & Selection of menu, food, etc. \\ \hline
			Picking team captain from a group. & Picking a team member. \\ \hline
			Picking $2$ colours, in order. & Picking $2$ colours, not in order. \\ \hline
			Picking $1$st, $2$nd and $3$rd winners. & Picking three winners. \\ \hline
		\end{tabular}
	\end{frame}
	
	\begin{frame}{Similarity in them.}
		\begin{enumerate}
			\item Permutation and combination are the ways to represent a group of objects by selecting them in a set and forming subsets.\pause \\ 
			\item They are an integral part of the field of counting in mathematics and these are extensively used in real life and have a lot of applications in business decision making.
		\end{enumerate}
	\end{frame}

\begin{frame}{References}
	\begin{thebibliography}{}
		\bibitem{1} https://byjus.com/maths/permutation-and-combination
		\bibitem{2} https://corporatefinanceinstitute.com/resources/knowledge/other/combination
		\bibitem{3} https://corporatefinanceinstitute.com/resources/knowledge/other/permutation
	\end{thebibliography}
\end{frame}
	
	\begin{frame}
		\begin{center}
			\includegraphics{thank}
		\end{center}
	\end{frame}
	
\end{document}