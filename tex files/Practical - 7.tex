\documentclass{article}
\usepackage{amsmath,amssymb,pst-plot,pstricks,graphicx}
\title{\textbf{Practical - 7} \\ Plotting of functions}
\author{Akriti Sengar \\ $2020705$}
\date{October $5^{\text{th}}$, 2021}
\begin{document}
	\maketitle
	\noindent \textbf{1. Graph the wildly oscillating function $y=\sin(1/x)$.}
	\begin{center}
		\begin{pspicture}(-0.25,-4.25)(7.5,4.25)
		\psset{xunit=3cm, yunit=3cm}
		\psaxes[showorigin=false]{->}(0,0)(0,-1.25)(2.25,1.25)
		\psplot[plotpoints=2500,linecolor=blue,linewidth=0.04]{0.025}{2}{1 x div RadtoDeg sin}
		\put(7,-0.5){$x$}
		\put(-0.5,4){$y$}
		\put(-0.5,-0.2){0}
		\put(3,3){$y=\sin \frac{1}{x}$}
		\end{pspicture}
	\end{center}
\newpage
	\noindent \textbf{2. Plot the graph of $x^2$ and $\sqrt{x}$.}
    	\begin{center}
    	\begin{pspicture}(-0.5,-1)(5,5.5)
    		\psset{xunit=4cm, yunit=4cm}
    		\psaxes{->}(0,0)(1.25,1.25)
    		\psset{plotpoints=500}
    		\psplot[linestyle=dotted,linecolor=pink,linewidth=0.13]{0}{1}{x sqrt}
    		\psplot[linestyle=dashed,linecolor=black,linewidth=0.06]{0}{1}{x 2 exp}
    		\put(5,-0.5){$x$}
    		\put(-0.5,5){$y$}
    		\put(2,3.75){$y=\sqrt{x}$}
    		\put(3,1.5){$y=x^2$}
    	\end{pspicture}
    \end{center}
    \newpage
	\noindent \textbf{3. Make a parametric plot of a lemniscate.}
    	\begin{center}
	\begin{pspicture}(-6,-8)(7,9)
		\psset{xunit=4cm, yunit=4cm}
		\psaxes[labels=none]{<->}(0,0)(-1.5,-1)(1.5,1)
		\parametricplot[plotpoints=350,arrows=->,fillstyle=solid,fillcolor=gray,linecolor=black,linewidth=0.09]{0}{360}{t cos 1 t sin 2 exp add div t sin t cos mul 1 t sin 2 exp add div}
		\put(5.8,-0.5){$x$}
		\put(0.3,3.85){$y$}
	\end{pspicture}
    \end{center}
\newpage	
	\noindent \textbf{4. Plot the graph of $f(x) = \begin{cases} 
			x^2, \quad x \leq 0 \leq 2 \\ 
			-x^2, \quad -2 \leq x < 0
		\end{cases}$}
    	\begin{center}
	\begin{pspicture}(-6,-8)(8,10)
		\psset{xunit=2cm, yunit=2cm}
		\psaxes{<->}(0,0)(-3,-4)(3,4)
		\psset{plotpoints=2000}
		\psplot[linecolor=pink,linewidth=0.09]{0}{2}{x 2 exp}
		\psplot[linecolor=black,linewidth=0.09]{-2}{0}{-1 x 2 exp mul}
		\put(5.9,-0.5){$x$}
		\put(-0.5,7.9){$y$}
		\put(3,3){$f(x)=x^2$}
		\put(-5,-4){$f(x)=-x^2$}
	\end{pspicture}
\end{center}
\newpage	

	\noindent \textbf{5. Plot $y = \sin x$ and $y = \cos x$ on the same coordinate system, for $0 \leq z \leq 2\pi$. Show the sine function as a solid curve and cosine function as a dotted curve.}
    	\begin{center}
	\begin{pspicture}(0,6)(6,4)
		\psaxes[ticks=none,labels=none]{<->}(0,0)(-2,-3)(8,4)
		\psplot[linecolor=pink,linewidth=0.1]{0}{6.28}{x RadtoDeg sin}
		\put(-0.3,-0.3){0}
		\put(2.9,-0.4){$2\pi$}
		\put(6.3,-0.4){$2\pi$}
		\put(-0.32,0.8){$1$}
		\psplot[linestyle=dashed,linecolor=black,linewidth=0.1]{0}{6.28}{x RadtoDeg cos}
		\put(1.8,1.3){$y=\sin{x}$}
		\put(2,-1.6){$y=\cos{x}$}
		\put(8.2,0){$x$}
		\put(-0.4,4){$y$}
	\end{pspicture}
\end{center}
\newpage
	\noindent \textbf{6. Plot $y = \sqrt{x} \sin (1/x)$, for $0 < x \leq 2$. On the same coordinate system, plot the functions $y = \sqrt{x}$ and $y = -\sqrt{x}$, for $0 \leq x \leq 2$, with these functions shown as dotted curves.}
    	\begin{center}
	\begin{pspicture}(6,6)
		\psset{xunit=3cm, yunit=2cm}
		\psaxes{<->}(0,0)(-1,-2)(3,2)
		\psplot[linestyle=dashed,linecolor=pink,linewidth=0.09,plotpoints=500]{0}{2}{x sqrt}
		\psplot[linestyle=dashed,linecolor=black,linewidth=0.09,plotpoints=500]{0}{2}{0 x sqrt sub}
		\psplot[plotpoints=500]{0.001}{2}{x sqrt 1 x div RadtoDeg sin mul}
		\put(9.2,0){$x$}
		\put(-0.4,3.8){$y$}
		\put(3.8,2.8){$y=\sqrt{x}$}
		\put(3.6,1){$y=\sqrt{x}\sin(1/x)$}
		\put(3.6,-3.2){$y=-\sqrt{x}$}
	\end{pspicture}
\end{center}
\newpage	
	\noindent \textbf{7. Plot the cardioid given by the parametric equations. \[x = \cos t(1-\cos t)\] \[y = \sin t(1-\cos t)\] $0 \leq t \leq 2 \pi$}
	    	\begin{center}
		\begin{pspicture}(-3,-4)(3,4)
			\psaxes[ticks=none,labels=none]{<->}(0,0)(-3,-3)(2,3)
			\parametricplot[linecolor=green,linewidth=0.09]{0}{360}{t cos 1 t cos sub mul t sin 1 t cos sub mul}
			\put(-2.2,1.5){Cardioid}
			\put(2.2,0){$x$}
			\put(-0.4,3){$y$}
		\end{pspicture}
	\end{center}
\end{document}